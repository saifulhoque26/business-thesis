\section{Conclusion}
%What will the expected result be?
In this thesis I have identified the key features that the users would like to see in our Fog Computing solution. I have also found out some features we envisioned in our early plan is not so desirable right now among the users. Hence, this thesis will guide us to develop a better solution by adding only desirable features in the next iteration of the product. Obviously, I will run more Build-Measure-Learn feedback loop from `lean startup' to test the next releases. However, right now this thesis have given me great feedbacks to develop what users' need and to create the best Fog computing solution in the market.

I think both design thinking and lean startup are great strategy to discover a customer segment's problem and develop a desirable product with minimum waste of resource. Selecting either one or both methods depends on the idea and development status. If someone uses only design thinking, then he might have a great solution that nobody wants. Whereas if someone uses only lean startup methods it might take more iteration to develop a product that solves most users' problem. Hence using design thinking to some extent in combination with lean startup would be an ideal feedback collecting strategy in my honest opinion. 



%will improve the business model for different machine-to-machine (M2M) or Internet of Things (IoT) applications. 

%For example:  You are launching a new startup (tech or non-tech).  If you begin by conducting some of the ethnographic studies or user research (design thinking style) to discover a customer segment's problem.  Then you utilize some rapid experimentation from Lean Startup to test that what you heard, observed or inferred from the customer interactions was actually true.

%If you had only used design thinking techniques, you could end up with a really incredible solution that you can't get customers to buy.   If you had only used lean startup, you might be in a position where it takes you MANY iterations to figure out what problems customers actually have and be able to market to them effectively.
